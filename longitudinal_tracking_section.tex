\subsection{Longitudinal Tracking and Risk Forecasting}

The system implements comprehensive longitudinal tracking to monitor patients' cardiac health over time, capturing the progression of risk factors and enabling temporal pattern analysis. This longitudinal approach is essential for early detection of deteriorating cardiac function and timely intervention.

\subsubsection{Longitudinal Data Structure}

Patient data is organized in a temporal framework that preserves the chronological relationship between measurements:

\begin{equation}
P = \{(t_1, x_1), (t_2, x_2), \ldots, (t_n, x_n)\}
\end{equation}

Where $P$ represents a patient's longitudinal record, $t_i$ is the timestamp of the $i$-th measurement, and $x_i$ is the corresponding feature vector. The system handles irregular time intervals between measurements using temporal interpolation techniques.

\subsubsection{Temporal Feature Extraction}

From the longitudinal data, the system extracts temporal features that capture trends and patterns:

\begin{itemize}
    \item \textbf{Slope Features}: Rate of change in clinical parameters (e.g., NT-proBNP slope)
    \item \textbf{Volatility Features}: Variability in measurements over time
    \item \textbf{Frequency Features}: Periodicity of abnormal events
    \item \textbf{Sequence Features}: Patterns of parameter changes preceding adverse events
\end{itemize}

These temporal features are calculated using:

\begin{equation}
\text{slope}(x, t_{i-k:i}) = \frac{\sum_{j=i-k}^{i} (t_j - \bar{t})(x_j - \bar{x})}{\sum_{j=i-k}^{i} (t_j - \bar{t})^2}
\end{equation}

\begin{equation}
\text{volatility}(x, t_{i-k:i}) = \sqrt{\frac{1}{k} \sum_{j=i-k}^{i} (x_j - \bar{x})^2}
\end{equation}

\subsubsection{Risk Forecasting Model}

The temporal forecasting model projects future heart failure risk based on current status and historical trends:

\begin{equation}
r_{t+\Delta t} = \alpha(\Delta t) \cdot f(x_t) + \beta(\Delta t) \cdot g(x_{t-k:t}) + (1-\alpha(\Delta t)-\beta(\Delta t)) \cdot h(x_t, \Delta t)
\end{equation}

Where:
\begin{itemize}
    \item $r_{t+\Delta t}$ is the predicted risk at time $t+\Delta t$
    \item $f(x_t)$ is the current risk prediction based on the hybrid model
    \item $g(x_{t-k:t})$ is a trend extrapolation function using historical data
    \item $h(x_t, \Delta t)$ is a time-dependent risk adjustment function
    \item $\alpha(\Delta t)$ and $\beta(\Delta t)$ are time-horizon-dependent weights
\end{itemize}

The weights are calculated as:

\begin{equation}
\alpha(\Delta t) = \frac{1}{1 + \gamma \cdot \Delta t}
\end{equation}

\begin{equation}
\beta(\Delta t) = \frac{\gamma \cdot \Delta t}{1 + \gamma \cdot \Delta t} \cdot \frac{\text{trend\_confidence}}{\text{trend\_confidence} + \text{base\_confidence}}
\end{equation}

Where $\gamma$ is a decay parameter that controls how quickly the current prediction's influence diminishes over time.

\subsubsection{Confidence Interval Calculation}

For each forecasted risk value, the system calculates confidence intervals based on historical volatility:

\begin{equation}
\text{CI}_{t+\Delta t} = [r_{t+\Delta t} - 1.96 \cdot \sigma \cdot \sqrt{\Delta t}, r_{t+\Delta t} + 1.96 \cdot \sigma \cdot \sqrt{\Delta t}]
\end{equation}

Where $\sigma$ is the historical volatility of risk predictions, calculated as the standard deviation of prediction errors over similar time horizons.

\subsubsection{Intervention Impact Modeling}

The system models the impact of interventions on future risk trajectories:

\begin{equation}
r_{t+\Delta t}^{I} = r_{t+\Delta t} - \sum_{i} \text{impact}_i \cdot (1 - e^{-\lambda_i \cdot \Delta t})
\end{equation}

Where $r_{t+\Delta t}^{I}$ is the risk with intervention, $\text{impact}_i$ is the maximum expected impact of intervention $i$, and $\lambda_i$ is the rate parameter controlling how quickly the intervention takes effect.
