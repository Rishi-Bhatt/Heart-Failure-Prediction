\subsubsection{Counterfactual Explanation Evaluation}

We evaluated the counterfactual explanation engine through both technical validation and clinical utility assessment:

\paragraph{Technical Validation}

We assessed the counterfactual generation algorithm using:

\begin{itemize}
    \item \textbf{Proximity}: How close counterfactuals are to the original data point
    \item \textbf{Sparsity}: The number of features changed in counterfactuals
    \item \textbf{Validity}: Whether counterfactuals actually achieve the desired prediction change
    \item \textbf{Plausibility}: Whether counterfactuals respect clinical constraints
\end{itemize}

\begin{table}[ht]
  \vspace{2mm}
  \centering
  \resizebox{0.95\columnwidth}{!}{%
    \begin{tabular}{|c|c|c|c|c|}
      \hline
      \textbf{Method} & \textbf{Proximity} & \textbf{Sparsity} & \textbf{Validity} & \textbf{Plausibility} \\
      \hline
      Our Method & 0.82 & 3.2 & 98\% & 96\% \\
      Unconstrained & 0.91 & 2.4 & 99\% & 67\% \\
      Rule-based only & 0.74 & 4.1 & 92\% & 98\% \\
      \hline
    \end{tabular}%
  }
  \vspace{2mm}
  \caption{Technical evaluation of counterfactual generation methods}
  \label{tab:counterfactual-tech-eval}
\end{table}

Our method achieved a strong balance between proximity and plausibility, ensuring that counterfactuals were both similar to the original data point and clinically realistic.

\paragraph{Clinical Utility Assessment}

We conducted a user study with 20 clinicians to evaluate the utility of counterfactual explanations:

\begin{table}[ht]
  \vspace{2mm}
  \centering
  \resizebox{0.95\columnwidth}{!}{%
    \begin{tabular}{|c|c|c|c|}
      \hline
      \textbf{Aspect} & \textbf{Rating (1-5)} & \textbf{Usefulness} & \textbf{Preference} \\
      \hline
      Clinical Relevance & 4.6 & 92\% & - \\
      Actionability & 4.3 & 85\% & - \\
      Understandability & 4.5 & 90\% & - \\
      \hline
      Counterfactuals & - & - & 75\% \\
      Feature Importance & - & - & 15\% \\
      Rule Explanations & - & - & 10\% \\
      \hline
    \end{tabular}%
  }
  \vspace{2mm}
  \caption{Clinical evaluation of counterfactual explanations}
  \label{tab:counterfactual-clinical-eval}
\end{table}

Clinicians rated counterfactual explanations highly for clinical relevance, actionability, and understandability. When asked to choose between different explanation methods, 75\% preferred counterfactual explanations over traditional feature importance or rule-based explanations.

\paragraph{Patient Case Studies}

We analyzed the impact of counterfactual explanations on treatment planning for 50 high-risk patients. In 78\% of cases, clinicians modified their treatment plans after reviewing the counterfactual scenarios, with changes including:

\begin{itemize}
    \item More aggressive blood pressure targets in 42\% of cases
    \item Addition of lifestyle interventions in 65\% of cases
    \item Modification of medication regimens in 53\% of cases
    \item More frequent monitoring in 38\% of cases
\end{itemize}

These results demonstrate that counterfactual explanations not only enhance understanding but also influence clinical decision-making in meaningful ways.
